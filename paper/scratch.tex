\paragraph{Extreme Distortion}
A nuanced problem arises in the Lambert Azimuthal projection, and the Hammer
  variants which approach it.
When a polygon of nonzero width is projected onto the edge of the map, it
  begins to distort into a crescent shape, eventually becoming a half-circle.
In itself, this can be faithfully rendered by interpolating the polygon.
However, in certain cases the splitting line goes down the center of the
  polygon, and we get as a result two opposing half-circles, one on each side
  of the map.
The effect is that our interpolated path becomes a full circle, without large
  jumps or visible discontinuities, as the location of the jump is at the pole.
Since we naively fill any polygon we render, the visual result is a colored
  circle overlaying the rest of the map.
This is relevant as Antarctica distorts into the bottom of the map, or as
  continents distort off of the sides (e.g., South America or Africa).

We address this problem by querying the projection regarding whether a
  proposed path will undergo extreme distortion.
If every vertex in the path is beyond a threshold, the path is not rendered.
This causes polygons near the edge of the map to disappear after a certain
  point; however, by then they are thin enough that the practical effect is
  minimal.

Importantly, we note that the proof-of-concept implementation does not handle
  this case, despite having the benefit of hiding polygons as they exit the
  viewport.
See \reffig{reference_bug}, or center the demo at zoom level 2.1 and coordinates
  $(100W, 65S)$.

\Fig{img/reference_bug.png}{0.25}{reference_bug}{
  A snapshot of the reference implementation from \cite{key:2012jenny-maps},
  found at
  \url{http://cartography.oregonstate.edu/demos/CompositeMapProjection/}.
  The coordinates are $(100W, 65S)$ at a zoom of 2.1.
  Note that Russia has distorted around the map, causing polygons to close
  themselves in front of the globe.
}
