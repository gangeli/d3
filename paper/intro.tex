\Section{intro}{Introduction}
\Fig{img/screenshot.png}{0.25}{screenshot}{
  A screenshot of our demo. The top screen shows the view that the user of the
  map projection would see; the bottom screen shows the projection as it would
  look from a global perspective.
  The particular view being shown is an interpolated projection between
  Albers Conic and Mercator, centered on San Francisco.
}
A significant challenge in selecting an appropriate map projection, and
  parametrization for that projection, for visualizing geographic data.
As no projection is good in every aspect, competing criteria include:

\begin{itemize}
  \item \textbf{Equal area}: A projection should preserve the areas of a
        polynomial. For instance, the Mercator projection presents inaccurate
        areas near the poles. In contrast, projections such as the Hammer,
        Lambert Azimuthal, or the Albers Conic projection preserve area.
  \item \textbf{Conformal}: A projection should preserve local angles.
        Mercator is an instance of such a projection; in contrast to any of the
        equal-area projections, which must necessarily distort angles somewhere.
  \item \textbf{Equidistant}:  Distances are preserved from a standard
        reference point. Although none of the projections in this paper have
        this property, an example of the projection is the azimuthal
        equidistant projection of the UN logo.
\end{itemize}

In addition, a projection should take into account to following criteria:

\begin{itemize}
  \item \textbf{Surface}: The choice of projecting onto, e.g., a cylinder versus
        a cone is often relevant. For example, showing the entire globe on a
        cone or other uncommon projection may serve to confuse the user more
        then enlighten them.
  \item \textbf{Aspect}: Often it is beneficial to view the map from an oblique
        angle; for example, for viewing areas near the poles.
  \item \textbf{Integration with existing services}: In particular, many of the
        maps on the Internet use the Mercator projection, including importantly
        Google Maps and OpenStreetMap. It's beneficial to be able to interface
        with these services.
\end{itemize}

A recent paper by Bernhard Jenny \cite{key:2012jenny-maps} implemented an
  adaptive projection, which attempts to balance the criteria above
  depending on the user's absolute latitude and the scale at which they
  view the map.
More details can be found in the Previous Work section.

We implement this paper as a proposed new projection in D3
  (see \reffig{screenshot}), and extend it to
  fit the production-ready standard and level of flexibility expected of D3.
In particular, we address two key challenges:

\paragraph{Wrapping}
The shape definitions used to render the globe come already split at the date
  line.
However, as we're proposing a map which has far greater flexibility in its
  viewport; and, should be able to handle user-defined polygons.
In general these may wrap around the edge of the map, causing artifacts to
  appear on the projected map.
We address this problem, described in more detail later, by interpolating paths
  and robustly detecting when such wrapping occurs.
The reference implementation handles wrapping for the small-scale projections,
  but opts to hide invisible polygons in most cases.

\paragraph{Extreme Distortion}
Certain projections exhibit extreme distortion on the edges of the map.
Particularly when taken jointly with wrapping, the effect is that there are
  occasionally phenomena which cannot be solved with interpolation alone.
The reference implementation does not handle this case.

In addition to these extensions, we implemented smooth animation between
  local points on a map.
The motivation is to give users context between two local areas, allowing them
  to orient themselves at a global, or near global scale, before transitioning
  to another local viewport.
In addition to being a useful feature in its own right, this showcases the
  capability and applicability of the projection, in that it requires accurate
  projections at many scales, and requires the efficiency to transition
  between them at a high frame rate.
