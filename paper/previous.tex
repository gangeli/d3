\Section{previous}{Previous Work}
The bulk of our project comes from the reimplimentation of a recent paper
  \cite{key:2012jenny-maps}; we summarize the approach here.

Much of the insight from Jenny's paper, in turn, stems from an earlier work
  by Snyder \cite{key:1987snyder-maps}, outlining a decision tree for which
  projections to use depending on the scale and location of the projection
  center.
In particular, Snyder proposed the following projections for different scales.
The projections used in our approach are distinguished in \textit{italics}

\begin{itemize}
  \item \textbf{World} \textit{Hammer}, Mollweide, Eckert IV or VI,
        McBryde, Boggs Eumorphic, Sinusoidal, or miscellaneous pseudocylindricals.
  \item \textbf{Hemisphere} \textit{Lambert Azimuthal}
  \item \textbf{Smaller Areas} We focus on square and landscape maps, for which
        Snyder proposes \textit{Lambert Cylindrical}, \textit{Albers Conic},
        and \textit{Lambert Azimuthal}.
\end{itemize}

Jenny outlines the criteria being optimized, which we in turn inherit:

\begin{itemize}
  \item \textbf{Distortion} An emphasis is placed on equal-area projections,
        particularly at large scales.
  \item \textbf{Graticules} Graticules are constrained by the
        following criteria:
        (1) They should be outward-pointing and not unusually-shaped;
        (2) The equator should be a straight line when the map is centered on it;
        (3) Straight longitude and latitude graticules are prefered where possible;
        and
        (4) Straight meridians should be used in polar views.
  \item \textbf{Esthetics} Esthetically, maps with eliptical borders are
        prefered to those with rectangular shapes.
  \item \textbf{Continuity} There should be continuity in the transitions
        between projections -- this is particulary important to our application,
        to help with smooth animations.
\end{itemize}

\FigStar{img/projections.png}{0.4}{jenny}{
A summary of projections used at different scales and absolute latitudes.
The $x$ axis denotes scale; the value corresponds to inverse fraction of the
  map the current viewport occupies.
Thus, at an $x$ value of 5, the viewport is $\frac{1}{5}$ the size of the map.
The $y$ axis corresponds to the absolute latitude of the center of the map;
  0 lies along the equator, which 90 is at the poles.
}

The final decision tree arrived at by the paper is summarized in
  \reffig{jenny}, and described in more detail below.
The details of the projections used are summarized in the next section.

\paragraph{World and Hemisphere Views}
At the world and hemisphere views, the paper suggests the Hammer and Lambert
  Azimuthal projections.
These are, in fact, two instances of the same generalized projection;
  interpolation is done smoothly between them by varying the hyperparameter
  which differentiates the projections.
At extreme latitudes, a polar aspect is used.
This results in the initially strange, but actually fairly useful phenomena
  where the map centers and distorts around a pole.
For example, the relatively small distance between North America and Europe
  over the North pole becomes far more salient.

\paragraph{Middle Views}
Most of the differentiation in projections come from the middle scales -- on the
  order of continents or countries.
In particular, near the equator we select the Lambert Cylindrical projection,
  optimizing for straight graticules.
Near the equator, this transitions seamlessly from Lambert Azimuthal.
In the majority of the middle latitudes, we use the Albers Conic projection.
Since we are sufficiently zoomed in, we do not see the esthetically unappealing
  properties of the conic projection; furthermore, the transition from
  Lambert Azimuthal is fairly painless.
Lastly, at extreme latitudes we maintain the Lambert Azimuthal projection.

\paragraph{Local Views}
A key practical consideration, as mentioned above, is to allow the adaptive
  projection to interface with other web services.
Importantly, to interface with tile-based services such as Google Maps and
  OpenStreetMap.
We would thus like to transition to a Mercator projection, which, in addition,
  preserves angle and shape accuracy for the small objects we are likely to
  see at this scale.
The transition is done by interpolating between the middle scale projections
  and Mercator with a simple linear interpolation.

We proceed to discuss the details of each projection in greater depth.
