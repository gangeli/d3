\Section{projections}{Projections Used}
Some of the projections used in the previous sections are special cases of
  generalized projections, or reduce to each other in limiting cases.
In this section we review each of the projections, and their relations to
  each other.
In our discussion, we will use $\lambda$ to denote the longitude and $\phi$
  to denote the latitude of the unprojected point.
The projected coordinates are denoted by $x$ and $y$.

\Subsection{hammer}{Generalized Hammer}
A generalized formula can be used to unify both the Hammer and the Lambert
  Azimuthal projections, given below:

\def\nu{ \sqrt{1 + \cos \phi \cos (\lambda / B)} }
\begin{align}
x &= \frac{ B \sqrt{2} \cos \phi \sin (\lambda / B) } { \nu } \\
y &= \frac{ \sqrt{2} \sin \phi }{ \nu }
\end{align}

The free variable $B$ can be used to select between the Hammer projection
  ($B = 2$) or the Lambert Azimuthal projection ($B = 1$).
Importantly, the parameter can also be varied $1 \leq B \leq 2$ to interpolate
  smoothly between the two projections.

\paragraph{Rotation Of Origin}
For these projections, it becomes important to rotate the earth such that it is
  centered on the center of the viewport.
While the longitudinal rotation is trivial, rotating towards the poles requires
  some sphirical geometry.
Given a rotated longitude $\lambda$, a latitude $\phi$, and a delta $\delta$ by
  which we would like to move, we compute our new longitude and latitude as:

\def\cosdelta{ \cos \delta }
\def\sindelta{ \sin \delta }
\def\clat{ \cos \phi }
\def\x{ \cos \lambda \clat }
\def\y{ \sin \lambda \clat }
\def\z{ \sin \phi }
\def\k{ \x \sindelta + \z \cosdelta }
\begin{align}
\lambda' &= \tan^{-1} \frac{\y}{ \x \cosdelta - \z \sindelta } \\
\phi' &= \sin^{-1} \k
\end{align}

Note that the argument to $\sin^{-1}$ in $\phi'$ is constrained to be between
  -1 and 1 in practice.

\Subsection{albers}{Albers Conic}
The albers conic projection, given an origin point $(\phi_0, \lambda_0)$
  and parameterized by standard parallels $\phi_l$ and $\phi_u$,
  is defined by the formul\ae\ below \cite{key:weisstein-albers}:

\def\th{ n \left( \lambda - \lambda_0 \right) }
\begin{align}
x &= \rho \sin ( \th ) \\
y &= \rho_0 - \rho \cos ( \th )
\end{align}

Where:

\begin{align*}
n &= \frac{1}{2} (\sin \phi_l + \phi_u) \\
C &= \cos^2 \phi_l + 2n\sin \phi_l \\
\rho &= \frac{
           \sqrt{ C - 2n \sin \phi }
         } { n } \\
\rho_0 &= \frac{
           \sqrt{ C - 2n \sin \phi_0 }
         } { n }
\end{align*}

Note that the Albers Conic projection can be coaxed into the Lambert Azimuthal
  and Lambert Cylindrical projections by changing the location of the standard
  parallels.

\Subsection{lambert}{Lambert Cylindrical}
Perhaps the simplest projection, the Lambert Cylindrical projection is defined
  as:

\begin{align}
x &= \lambda - \lambda_0  \\
y &= \sin \phi
\end{align}

